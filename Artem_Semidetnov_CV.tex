\documentclass[10pt, letterpaper]{article}

% Packages:
\usepackage[
    ignoreheadfoot, % set margins without considering header and footer
    top=2 cm, % seperation between body and page edge from the top
    bottom=2 cm, % seperation between body and page edge from the bottom
    left=2 cm, % seperation between body and page edge from the left
    right=2 cm, % seperation between body and page edge from the right
    footskip=1.0 cm, % seperation between body and footer
    % showframe % for debugging 
]{geometry} % for adjusting page geometry
\usepackage[explicit]{titlesec} % for customizing section titles
\usepackage{tabularx} % for making tables with fixed width columns
\usepackage{array} % tabularx requires this
\usepackage[dvipsnames]{xcolor} % for coloring text
\definecolor{primaryColor}{RGB}{0, 79, 144} % define primary color
\usepackage{enumitem} % for customizing lists
\usepackage{fontawesome5} % for using icons
\usepackage{amsmath} % for math
\usepackage[
    pdftitle={Artem Semidetnov's CV},
    pdfauthor={Artem Semidetnov},
    pdfcreator={LaTeX with RenderCV},
    colorlinks=true,
    urlcolor=primaryColor
]{hyperref} % for links, metadata and bookmarks
\usepackage[pscoord]{eso-pic} % for floating text on the page
\usepackage{calc} % for calculating lengths
\usepackage{bookmark} % for bookmarks
\usepackage{lastpage} % for getting the total number of pages
\usepackage{changepage} % for one column entries (adjustwidth environment)
\usepackage{paracol} % for two and three column entries
\usepackage{ifthen} % for conditional statements
\usepackage{needspace} % for avoiding page brake right after the section title
\usepackage{iftex} % check if engine is pdflatex, xetex or luatex

% Ensure that generate pdf is machine readable/ATS parsable:
\ifPDFTeX
    \input{glyphtounicode}
    \pdfgentounicode=1
    \usepackage[T1]{fontenc}
    \usepackage[utf8]{inputenc}
    \usepackage{lmodern}
\fi

\usepackage[default, type1]{sourcesanspro} 

% Some settings:
\AtBeginEnvironment{adjustwidth}{\partopsep0pt} % remove space before adjustwidth environment
\pagestyle{empty} % no header or footer
\setcounter{secnumdepth}{0} % no section numbering
\setlength{\parindent}{0pt} % no indentation
\setlength{\topskip}{0pt} % no top skip
\setlength{\columnsep}{0.15cm} % set column seperation
\makeatletter
\let\ps@customFooterStyle\ps@plain % Copy the plain style to customFooterStyle
\patchcmd{\ps@customFooterStyle}{\thepage}{
    \color{gray}\textit{\small Artem Semidetnov - Page \thepage{} of \pageref*{LastPage}}
}{}{} % replace number by desired string
\makeatother
\pagestyle{customFooterStyle}

\titleformat{\section}{
    % avoid page braking right after the section title
    \needspace{4\baselineskip}
    % make the font size of the section title large and color it with the primary color
    \Large\color{primaryColor}
}{
}{
}{
    % print bold title, give 0.15 cm space and draw a line of 0.8 pt thickness
    % from the end of the title to the end of the body
    \textbf{#1}\hspace{0.15cm}\titlerule[0.8pt]\hspace{-0.1cm}
}[] % section title formatting

\titlespacing{\section}{
    % left space:
    -1pt
}{
    % top space:
    0.3 cm
}{
    % bottom space:
    0.2 cm
} % section title spacing

% \renewcommand\labelitemi{$\vcenter{\hbox{\small$\bullet$}}$} % custom bullet points
\newenvironment{highlights}{
    \begin{itemize}[
        topsep=0.10 cm,
        parsep=0.10 cm,
        partopsep=0pt,
        itemsep=0pt,
        leftmargin=0.4 cm + 10pt
    ]
}{
    \end{itemize}
} % new environment for highlights

\newenvironment{highlightsforbulletentries}{
    \begin{itemize}[
        topsep=0.10 cm,
        parsep=0.10 cm,
        partopsep=0pt,
        itemsep=0pt,
        leftmargin=10pt
    ]
}{
    \end{itemize}
} % new environment for highlights for bullet entries


\newenvironment{onecolentry}{
    \begin{adjustwidth}{
        0.2 cm + 0.00001 cm
    }{
        0.2 cm + 0.00001 cm
    }
}{
    \end{adjustwidth}
} % new environment for one column entries

\newenvironment{twocolentry}[2][]{
    \onecolentry
    \def\secondColumn{#2}
    \setcolumnwidth{\fill, 4.5 cm}
    \begin{paracol}{2}
}{
    \switchcolumn \raggedleft \secondColumn
    \end{paracol}
    \endonecolentry
} % new environment for two column entries

\newenvironment{threecolentry}[3][]{
    \onecolentry
    \def\thirdColumn{#3}
    \setcolumnwidth{1 cm, \fill, 4.5 cm}
    \begin{paracol}{3}
    {\raggedright #2} \switchcolumn
}{
    \switchcolumn \raggedleft \thirdColumn
    \end{paracol}
    \endonecolentry
} % new environment for three column entries

\newenvironment{header}{
    \setlength{\topsep}{0pt}\par\kern\topsep\centering\color{primaryColor}\linespread{1.5}
}{
    \par\kern\topsep
} % new environment for the header

\newcommand{\placelastupdatedtext}{% \placetextbox{<horizontal pos>}{<vertical pos>}{<stuff>}
  \AddToShipoutPictureFG*{% Add <stuff> to current page foreground
    \put(
        \LenToUnit{\paperwidth-2 cm-0.2 cm+0.05cm},
        \LenToUnit{\paperheight-1.0 cm}
    ){\vtop{{\null}\makebox[0pt][c]{
        \small\color{gray}\textit{Last updated in January 2025}\hspace{\widthof{Last updated in January 2025}}
    }}}%
  }%
}%

% save the original href command in a new command:
\let\hrefWithoutArrow\href

% new command for external links:
\renewcommand{\href}[2]{\hrefWithoutArrow{#1}{\ifthenelse{\equal{#2}{}}{ }{#2 }\raisebox{.15ex}{\footnotesize \faExternalLink*}}}


\begin{document}
    \newcommand{\AND}{\unskip
        \cleaders\copy\ANDbox\hskip\wd\ANDbox
        \ignorespaces
    }
    \newsavebox\ANDbox
    \sbox\ANDbox{}

    \placelastupdatedtext
    \begin{header}
        \fontsize{30 pt}{30 pt}
        \textbf{Artem Semidetnov}

        \vspace{0.3 cm}

        \normalsize
        \mbox{{\footnotesize\faMapMarker*}\hspace*{0.13cm}Saint-Petersburg}%
        \kern 0.25 cm%
        \AND%
        \kern 0.25 cm%
        \mbox{\hrefWithoutArrow{mailto:artemsemidetnov@gmail.com}{{\footnotesize\faEnvelope[regular]}\hspace*{0.13cm}artemsemidetnov@gmail.com}}%
        \kern 0.25 cm%
        \AND%
        \kern 0.25 cm%
        \mbox{\hrefWithoutArrow{tel:+7-921-363-89-09}{{\footnotesize\faPhone*}\hspace*{0.13cm}8 (921) 363-89-09}}%
    \end{header}

    \vspace{0.3 cm - 0.3 cm}


    \section{Education}



        
        \begin{threecolentry}{\textbf{BS}}{
            2021 – 2025
        }
            \textbf{Saint-Petersburg State University}, Mathematics
        \end{threecolentry}

        \vspace{0.2 cm}

        \begin{threecolentry}{\textbf{BS}}{
            2022 – 2024
        }
            \textbf{Neapolis University Of Pafos}, Applied Computer Science JetBrains Program
        \end{threecolentry}


    
    \section{Publications}



        
        \begin{samepage}
            \begin{twocolentry}{
                May 2021
            }
                \textbf{\href{https://arxiv.org/pdf/2106.00095}{On the geometry of free nilpotent groups}}

                \vspace{0.10 cm}

                \mbox{\textbf{\textit{Artem Semidetnov}}}, \mbox{Ruslan Magdiev}
                \vspace{0.10 cm}

        \href{https://doi.org/https://doi.org/10.48550/arXiv.2106.00095}{https://doi.org/10.48550/arXiv.2106.00095}
            \end{twocolentry}
        \end{samepage}


    
    \section{Awards and Scholarships}



        
        \begin{onecolentry}
            Scholarship by "Rodnye Goroda" (a social investment program of PJSC "Gazprom neft"), 2023-2024
        \end{onecolentry}

        \vspace{0.2 cm}

        \begin{onecolentry}
            JetBrains Scholarship in Neapolis University
        \end{onecolentry}


    
    \section{Teaching Experience}



        
        \begin{twocolentry}{
            Sochi, Russia

        Apr 2024 – May 2024
        }
            \textbf{Sirius educational center}, Teaching assistant on the course \href{https://sochisirius.ru/obuchenie/nauka/smena1783/8258}{"Groups of intermediate growth"}
        \end{twocolentry}


        \vspace{0.2 cm}

        \begin{twocolentry}{
            Saint-Petersburg, Russia

        2021 – 2023
        }
            \textbf{Laboratory for continuous mathematical education}, mathematics teacher for gifted students. I was teaching introduction to group theory.
        \end{twocolentry}


        \vspace{0.2 cm}

        \begin{twocolentry}{
            Saint-Petersburg, Russia

        June 2023
        }
            \textbf{Mathematics and Computer Science faculty program for prospective students}, Teaching assistant on the course "Braid and knot theory"
        \end{twocolentry}



    
    \section{Selected Talks}



        
        \begin{twocolentry}{
            Novosibirsk, Russia

        2021
        }
            \textbf{On the geometry of free nilpotent groups}
            \begin{highlights}
                \item Siberian summer conference
            \end{highlights}
        \end{twocolentry}


        \vspace{0.2 cm}

        \begin{twocolentry}{
            Saint-Petersburg, Russia
        }
            \textbf{\href{https://m.mathnet.ru/php/seminars.phtml?option_lang=rus\&presentid=34355}{"On the Poisson boundary of lamplighter groups"}}
            \begin{highlights}
                \item St. Petersburg Seminar on Representation Theory and Dynamical Systems
            \end{highlights}
        \end{twocolentry}


        \vspace{0.2 cm}

        \begin{twocolentry}{
            N.Novgorod, Russia

        2024
        }
            \textbf{\href{https://disk.yandex.ru/i/52gkQhVjjRravA}{"Twisting numbers on braid and Thompson’s groups"}}
            \begin{highlights}
                \item Topological Methods in Dynamics and Related Topics VII
            \end{highlights}
        \end{twocolentry}



    
    \section{Work Experience}



        
        \begin{twocolentry}{
            Pafos, Cyprus

        June 2024 – Sept 2024
        }
            \textbf{JetBrains}, Research Intern in HoTT and Dependent Types Lab
            \begin{highlights}
                \item I was developing the official library in the Arend language. I formalized different results in algebra and homotopy type theory, including following.
                \item Formalized Eckmann-Hilton argument, Eilenberg-Maclane spaces, Homologies of types.
                \item Formalized automorphisms of groups, Schur's Lemma, Maschke's Lemma, Group actions characterizations. (Some of these results are in the \href{https://arend-lang.github.io/2024/07/05/Arend-1.10.0-released.html}{1.10 release})
            \end{highlights}
        \end{twocolentry}


        \vspace{0.2 cm}

        \begin{twocolentry}{
            Pafos, Cyprus

        May 2023 – Sept 2023
        }
            \textbf{IPONWEB (acquired by Criteo)}, Machine Learning Intern
            \begin{highlights}
                \item Criteo has a ML tool that analyzes sites and produces word-2-vec representations. In IPONWEB I was trying to reverse-engineer the behaviour of this tool and analyze its possible applications.
            \end{highlights}
        \end{twocolentry}



    
    \section{Miscellaneous}

    \begin{onecolentry}
        \begin{highlightsforbulletentries}


        \item Finalist of 2020 Intel ISEF

        \item Intel ISEF alumni

        \item Winner of 2019 Baltic SEF, PDMI special prize in 2019 Baltic SEF

        \item 3rd team place in 2019 \href{https://www.itym.org}{International Tournament of Young Mathematicians} in Barcelona, Spain

        \item Organizer of the \href{https://indico.eimi.ru/category/102/}{Euler International Mathematical Institute's functional analysis seminar}

        \item English level C1 (IELTS 8.0/9, taken in 2020, 2024)

        \item Invited judge in Saint-Petersburg Tournament of Young Mathematicians (since 2021)

        \item Created mathematical problem for 2024 International Tournament of Young Mathematicians (\href{https://drive.google.com/file/d/1d4dqKrTKG6MI_cYQwnos8iDJWCuVIs4I/view}{10th in here}).

        \item Invited judge in International Tournament of Young Mathematicians 2024


        \end{highlightsforbulletentries}
    \end{onecolentry}


\end{document}